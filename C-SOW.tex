%1. A general description of the objective (for each defined task/activity);
%2. A detailed description of the approach to be taken to accomplish each defined task/activity;
%3. Identification of the primary organization responsible for task execution (prime, sub, team member, by name, etc.);
%4. The completion criteria for each task/activity - a product, event or milestone that defines its completion.
%5. Define all deliverables (reporting, data, reports, software, etc.) to be provided to the Government in support of the proposed research tasks/activities; AND
%6. Clearly identify any tasks/subtasks (prime or subcontracted) that will be accomplished on-campus at a university.

\noindent
All the tasks in Phases 1, 2, and 3 will be on-campus at UIUC, USC, and Columbia, unless specified otherwise.

\noindent
\textbf{Phase 1.}

\noindent
\textbf{Task 1: Architect the accelerator module.} 
Kim and Hwu at UIUC (prime-contractor) with Annavaram at USC (sub-contractor) will architect the proposed accelerator module.
This includes defining the instruction set and overall architecture of the accelerator and implementing the functional simulator of the accelerator module.
This task is completed when we can run test code using the functional simulator and deliver the items described below.
We will deliver the completed software code for the functional simulator, test code, and technical documents describing the instruction set and architecture with block diagrams, simulator implementation, and usage of the simulator.


\noindent
\textbf{Task 2: Model the processing pipeline.} 
Annavaram will model the microarchitecture of the proposed processing pipeline in software.
This includes (1) architecting coarse-grain reconfigurable execution pipeline for accelerating graph primitives requiring complex operations and (2) Trailbalzer..
% TODO: need to elaborate when the architecture is finalized

\noindent
\textbf{Task 3: Model the memory sub-system.}
Kim will model the microarchitecture of the proposed memory sub-system in software. 
This includes architecting (1) proposed custom SRAM-based on-chip caches private to each acceleration core; (2) a custom eDRAM-based on-chip cache shared by all the acceleration cores in a chip; (3) a custom memory controller; and (4) a prefetcher.

\noindent
\paragraph{Task 4: Model the communication sub-systems.}
Hwu will model the microarchitecture of the proposed intra- and inter-chip communication architectures. 
This includes architecting (1) on-chip network interfaces between accelerator cores and eDRAM-based shared cache; and (2) off-chip network interfaces amongst the host system and other accelerator chips.
% TODO: need to elaborate when the architecture is finalized


\noindent
\textbf{Tasks 2} and \textbf{3} are completed when the software code can successfuly run the test code derived from given benchmarks, provide the estimation of performance in terms of simulated cycles, and deliver the items below.
We will deliver the completed software code, test code, and technical documents describing the architecture with block diagrams and schematics, software implementation and usage of the software code.

\noindent
\textbf{Task 4: Design the eDRAM cache circuit.}
Seok at Columbia (sub-contractor)will design an eDRAM cache in a 28nm CMOS technology.
This includes the development of an eDRAM array circuit that (1) support spatial and temporal boosting; and (2) non-invasive thermal sensors and a dynamic thermal management scheme to ensure that the spatio and temporal boosting does not violate a thermal limit.
% TODO: Mingoo please check. Be mindful that we have limited space.

\noindent
\textbf{Task 5: Design the 1TBps link circuit.}
Hanumolu at UIUC (prime-contractor) will design a 1TBps link circuit in a 28nm CMOS technology.
This includes the development of circuit structures that employ (1) injection-locking and (2) clocking techniques that improve power and area efficiency. 
% TODO: Pavan please elaborate.

\noindent
\textbf{Tasks 4} and \textbf{5} are completed when HSPICE can successfully simulate the the designed eDRAM array and link circuits, and demonstrate the projected performance and power-efficiency improvements, respectively, 
We will deliver the completed link circuit and testbench netlists, performance parameters needed for the architectural simulation, and technical document describing the design and HSPICE simulation results.

\noindent
\paragraph{Task 6: Model the entire accelerator module.}
Hwu will model the microarchitecture of proposed intra- and inter-chip communication architectures 

and Hwu, Kim, and Annavaram will integrate the microarchitecture models of the processing pipeline (Task 2) and memory subsystem (Task 3) models with the communication architectures to model the microarchitecture of the entire accelerator module in software. 
When the simulator software code can successfuly run the test code derived from given benchmarks with the input set fitting into the on-package memory system comprised of 4$\times$ 8GB DiRAM4 dies, and provide the estimation of performance in terms of simulated cycles, this task is completed.
We will deliver the completed integrated simulator code, test code, and technical document describing the entire architecture with block diagrams and schematics, implementation in C++, and usage of the simulator.


\noindent
\textbf{Phase 2.}

\noindent
\textbf{Task 6: Implement the processing pipeline.} 
The Annavaram's team at USC will implement the modeled processing pipeline architecture.


\noindent
\textbf{Task 7: Implement the memory subsystem.} 
The Kim's team at UIUC will implement the modeled memory subsystem architecture.

\noindent
\textbf{Tasks 6} and \textbf{7} include the RTL designs in Verilog HDL, synthesis of the designs to generate gate-level netlists, and test of the synthesized netlists using a gate-level simulator and an FPGA development board.
When the synthesized netlists can successfuly run the test code used in Tasks 1 and 2 in an FPGA development board, these two tasks are completed.
We will deliver the completed Verilog HDL code, synthesized gate-level netlists, synthesis scripts, simulation scripts, and technical document describing the RTL design and expected operating frequency in the target 28nm technology.

\noindent
\textbf{Task 8: Physical-design the eDRAM cache.} 
Task 2.X: Cache physical design (layout) with fine-grained access and spatiotemporal boosting in a 28nm CMOS technology
Milestone 2.X.1: Demonstration of cache circuits based on the array developed in the Task 1.X. The demonstrated cache circuits can perform narrow- and wide-width data accesses from multiple non-contiguous memory locations. \underline{Seok (Columbia), Kim (UIUC)}
Completion criteria: functioning cache schematics, layout, and chip prototype
Approach: If non-contiguous memory locations are in multiple banks, we will parallelize the accesses to hide latency penalty. If non-contiguous memory locations are in the same banks, we will leverage the throughput improvement of the spatiotemporal boosting technique for minimizing latency penalty for accessing non-contiguous memory locations. We will maximize circuit portions that are not boosted such that we can use them as thermal buffer and extend the time and spatial range of supply voltage boosting. 


\noindent
\textbf{Task 9: Phsyical-design the 1TBps link.} 

\noindent
\textbf{Task 10: Implement the entire accelerator module.} 
The Hwu's team at UIUC will implement the proposed intra- and inter-chip communication architectures and integrate the implemented processing pipeline (Task 1) and memory subsystem (Task 2) block with the communication architectures 
This include the design in Verilog HDL, synthesis of the design to generate RTL, and test of the synthesized netlist using a gate-level simulator such as ModelSim and FPGA.
When the synthesized netlist can successfuly run the test code used in Task 5 in an FPGA development board, this task is completed.
We will deliver the completed Verilog HDL code, synthesized gate-level netlist, synthesis scripts, simulation scripts, and technical document describing the RTL design and expected operating frequency in the target 28nm technology.

\textbf{Task 11: Refine the accelerator architecture.} 


\noindent
\textbf{Task 11: Phsyical-design the accelerator chip.} 
The Seok's team at Columbia will ...
When the generated GDS2 file for the entire accelerator is submitted to the foundry, this task is completed.
We will deliver the completed GDS2 file.
Task 2.Y: Integration, physical design, and simulations of the accelerator chip in a 28nm CMOS technology. \underline{Seok (Columbia), Kim (UIUC), Hanumolu (UIUC)}
Completion criteria and deliverables: accelerator chip layout, functionality, performance, and power dissipation verified by post-layout simulations
Approach: We will integrate the layouts of our custom designed parts (pipelines, caches, links) and third-party IPs (a DRAM controller for Tezzaron's DiRAM). The accelerator will include on-chip testing circuitries including boundary scan-chains, on-chip clock generators, and built-in-self-test (BIST) blocks. We will design the connection structures in the accelerator chip for 2.5D integration. The post-layout netlist will be generated and tested in fast SPICE simulators. 


\noindent
\textbf{Task 12: Performing 2.5D the phsyical design of the accelerator chip.} 
The Seok's team at Columbia will ...
When the generated GDS2 file for the entire accelerator is submitted to the foundry, this task is completed.
We will deliver the completed GDS2 file.


\noindent
\textbf{Task 12: FPGA-prototype a single accelerator connected with a host system.}

We use .. FPGA development boards that can be connected to a host system to prototype

\noindent
\textbf{Task 13: Preparing a PCB board schematic for 16 accelerators.}
The board design will be subcontacted to Nallatech.
When 


\noindent
\textbf{Phase 3:}

\noindent
\textbf{Task : 2.5D-integrating an accelerator die with DiRAM dies.}

\noindent
\textbf{Task : Build.}



Task 3.X: Our accelerator chip fabrication and chip-level testing \underline{Seok (Columbia), Kim (UIUC), Hanumolu (UIUC), Annavaram (USC), Hwu (UIUC)}

   Completion criteria: fabricated chips, verified in the chip level

   Approach: We will perform intensive pre-silicon verifications for functional, thermal, voltage integrity, noise, and testability aspects. We will tape-out the accelerator chip in a 28nm CMOS technology. We will test the accelerator chip without DiRAMs. 

Task 3.2: 2.5D integration of our accelerator chip and memory stacks, packaging, and testing \underline{?}

   Completion criteria and deliverables: Functioning 2.5D integrated hardware of the accelerators and DiRAMs. 

   Approach:  We will integrate the accelerator chip with two DiRAMs using a silicon interposer. The silicon interposer that creates 2,000 connection to the accelerator chip per DiRAM will be fabricated by a company (XXXXX). The interposer will be either connected to a custom PCB directly or enclosed on a BGA type package. If necessary, we will mount a off-the-shelf cooler, potentially modified to fit, on top of the accelerator chip. 
