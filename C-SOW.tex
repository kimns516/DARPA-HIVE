%1. A general description of the objective (for each defined task/activity);
%2. A detailed description of the approach to be taken to accomplish each defined task/activity;
%3. Identification of the primary organization responsible for task execution (prime, sub, team member, by name, etc.);
%4. The completion criteria for each task/activity - a product, event or milestone that defines its completion.
%5. Define all deliverables (reporting, data, reports, software, etc.) to be provided to the Government in support of the proposed research tasks/activities; AND
%6. Clearly identify any tasks/subtasks (prime or subcontracted) that will be accomplished on-campus at a university.

\noindent
All the tasks in Phases 1, 2, and 3 will be performed on-campus at UIUC, USC, and Columbia, unless specified otherwise.

\noindent
\textbf{Phase 1.}

\noindent
\textbf{(Task-1) Architect the accelerator module.} 
Kim at UIUC (prime) and Annavaram at USC (sub) with Hwu at UIUC (prime) will define the instruction set and overall architecture of the accelerator and implement the functional simulator of the accelerator module in software.
This task is completed when the functional simulator can run test code and deliver the following items: 
(1) the completed software code for the functional simulator with the test code; and (2) technical documents depicting the instruction set, architecture diagrams, and simulator implementation with the simulator usage.


\noindent
\textbf{Model (Task-2) processing pipeline; (Task-3) memory sub-system; and (Task-4) communication sub-system.}
Annavaram, Kim, and Hwu will model the microarchitecture and performance of 
(Task-2) (1) coarse-grain reconfigurable execution pipelinel and (2) ...;
% TODO: need to elaborate when the architecture is finalized
(Task-3) (1) custom SRAM-based on-chip caches private to each acceleration core; (2) a custom eDRAM-based on-chip cache shared by all the acceleration cores in a chip; (3) a custom memory controller; and (4) a prefetcher.
(Task-4) (1) on-chip network interfaces between accelerator cores and eDRAM-based shared cache; and (2) off-chip network interfaces amongst the host system and other accelerator chips in software, respectively.
These tasks are completed, when the software code from the tasks can successfuly run test code derived from given benchmarks; provide performance in terms of cycles; and deliver the following items:
(1) the completed performance modeling software code with the test code; and (2) technical documents depicting microarchitecture diagrams and performance simulator implementation with the simulator usage.


\noindent
\textbf{Design (Task-5) the eDRAM cache and (Task-6) 1TBps link circuits.}
Seok at Columbia (sub) and Hanumolu at UIUC (prime) will design 
(Task-5) an eDRAM array circuit that (1) support spatial and temporal boosting; and (2) non-invasive thermal sensors and a dynamic thermal management scheme; and 
(Task 6) a 1TBps link circuit that employ (1) injection-locking; and (2) clocking techniques in a 28nm CMOS technology, respectively.
These tasks are completed when HSPICE can successfully simulate the netlists from the tasks, demonstrate the projected performance and power-efficiency improvements, and deliver the following items:
(1) the completed circuit schematics with testbench schematics; (2) performance parameters needed for the architectural simulation; and (3) technical document describing the design and HSPICE simulation results.


\noindent
\textbf{(Task-7) Model the entire accelerator module.}
Hwu, Kim, and Annavaram will integrating the models from Tasks 1 -- 4 to model the entire proposed accelerator module and and its performance in software. 
This task is completed when we the performance simulator can successfuly run test code derived from given benchmarks with the input set fitting into the 4$\times$ 8GB DiRAM4 dies; provide performance in terms of simulated cycles; and deliver the following items:
(1) the completed performance simulator software code with the test code; and (2) technical documents depicting architecture diagrams and performance simulator implementation with the simulator usage.


\noindent
\textbf{Phase 2.}

\noindent
\textbf{Implement (Task-8) processing pipeline; (Task-9) memory sub-system; and (Task-10) communication sub-system}
Annavaram, Kim, and Hwu will implement the processing pipeline, memory sub-system, and communication sub-system microarchitectures 
based on Tasks 1 -- 3, respectively.


\noindent
\textbf{(Task-11) Implement the entire accelerator module.}
Hwu, Kim, and Annavaram will integrating the processing pipeline, memory sub-system, and communcation sub-system from Tasks 8 -- 10 to implement the entire proposed accelerator module. 


\noindent
\textbf{Tasks 8}, \textbf{9}, \textbf{10}, and \textbf{11} include the RTL designs in Verilog HDL, synthesis of the designs to generate gate-level netlists, and test of the synthesized netlists using a gate-level simulator and an FPGA development board.
These four tasks are completed when the synthesized netlists can successfuly run the test code used in Tasks 1 -- 4 in an FPGA development board and deliver the following items:
(1) the completed Verilog HDL code with synthesized gate-level netlists, synthesis scripts, simulation scripts for each task; and (2) technical document describing the RTL design and expected operating frequency of the accelerator module in the target 28nm technology.


\noindent
\textbf{(Task-12) Refine the accelerator module architecture.} 
Hwu, Kim, and Annavaram will refine the architecture of the accelerator module and microarchitectures of accelerator components after analyzing the performance of the accelerator module with the architectural simulator refined by the synthesis results.
This task is completed when the inaccurate architectural models and their parameters are updated based on the synthesis results; the architecture and microarchitectures are improved based on the simulations using the updated architectural simulator; the RTL designs are updated accordingly; and deliver the following items:
(1) the improved architectural simulator software code, RTL designs, synthesized gate-level netlists, and synthesis scripts; and (2) technical document describing the improvement and updated performance projection.


\noindent
\textbf{Perform the physical design of (Task-13) eDRAM cache and (Task-14)  1TBps link circuits.} 
Seok and Hanumolu will perform the physical design of eDRAM cache circuits from Task 5 and 1TBps link circuits from Task 6, respectively.
These tasks are completed when HSPICE can successfully simulate the post-layout netlists from the tasks, demonstrate the projected performance and power-efficiency improvements, and deliver the following items:
(1) the completed GDS2 files; and (2) technical document describing the HSPICE post-layout simulation results.


\noindent
\textbf{(Task-15) Perform the physical design of the accelerator.} 
Seok at Columbia with Hanumolu at UIUC will design on-chip testing circuits including boundary scan-chains, on-chip clock generators and built-in-self-test (BIST) blocks, and perform their physical design.
Then, they will integrate the physical design of the on-chip testing circuits and the with that of the eDRAM cache, 1TBps link, synthesized processing pipeline, on-chip memory sub-system and communication sub-system blocks, and third-party hard IP blocks such as PCIe PHY and PLL.
Lastly, they will design the connection structures in the accelerator die for 2.5D integration. 
This task is completed when the DRC- and LVS-clean GDS2 is generated and submitted to the foundry, and the post-layout fast timing, power and signal integrity simulations meet our expectations.


\noindent
\textbf{Task 16: FPGA-prototype a single accelerator connected with a host system.}

We use .. FPGA development boards that can be connected to a host system to prototype

\noindent
\textbf{Task 17: Preparing a PCB board schematic for 16 accelerators.}
The board design will be subcontacted to Nallatech.
When 


\noindent
\textbf{Phase 3:}

\noindent
\textbf{Task : 2.5D-integrating an accelerator die with DiRAM dies.}

\noindent
\textbf{Task : Build.}



Task 3.X: Our accelerator chip fabrication and chip-level testing \underline{Seok (Columbia), Kim (UIUC), Hanumolu (UIUC), Annavaram (USC), Hwu (UIUC)}

   Completion criteria: fabricated chips, verified in the chip level

   Approach: We will perform intensive pre-silicon verifications for functional, thermal, voltage integrity, noise, and testability aspects. We will tape-out the accelerator chip in a 28nm CMOS technology. We will test the accelerator chip without DiRAMs. 

Task 3.2: 2.5D integration of our accelerator chip and memory stacks, packaging, and testing \underline{?}

   Completion criteria and deliverables: Functioning 2.5D integrated hardware of the accelerators and DiRAMs. 

   Approach:  We will integrate the accelerator chip with two DiRAMs using a silicon interposer. The silicon interposer that creates 2,000 connection to the accelerator chip per DiRAM will be fabricated by a company (XXXXX). The interposer will be either connected to a custom PCB directly or enclosed on a BGA type package. If necessary, we will mount a off-the-shelf cooler, potentially modified to fit, on top of the accelerator chip. 
