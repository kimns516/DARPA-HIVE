\noindent
%HIVES calls for creation of processor-to-memory and processor-to-processor communication interfaces that can support extraordinarily high bandwidth exceeding 1TB/s or equivalently 8Tb/s. 
GAMA's inter-tile communication interfaces can support bandwidth exceeding 1TB/s.
Providing such high bandwidth requires a total of either 800 lanes operating at 10Gb/s/lane or 320 lanes operating at 25Gb/s/lane.  
Such a large number of lanes %(even when spread across multiple chips) 
operating at high date rates poses many difficult challenges. 
For example, the total power consumption of the state-of-the-art 10mW/Gb/s links will be in excess of 80W~\cite{krishnaswamy2013bandwidth}, 
far exceeding the 20W power constraint.
%Second, interference and coupling resulting from operating a large number of closely spaced parallel lanes also severely limits performance. 
%Increasing the per lane data rate in an attempt to reduce the total number of lanes (for example, 50Gb/s/lane reduces the number of lanes to 160) is counter productive because it degrades power efficiency. 
%To elucidate this tradeoff further, increasing the data rate by a factor of 2 increases the channel loss by at least 6dB and circuit operating frequency by a factor of 2, both of which potentially increase the total by much more than 2$\times$. 
In view of this, we seek to architect new low-loss channels and connectors and improve the per-lane energy efficiency to the order of 1mW/Gb/s. 
%Because the GAMA tiles are going to be designed to minimize processor idle time, dynamic power management techniques such as those described in a prior study (Anand et al., JSSC 2015) cannot be used to improve power efficiency.  Instead, the techniques that improve power efficiency at peak data are needed.


Large portion of the power consumed in multi-lane transceivers is used to perform equalization, clock generation/recovery/distribution and serializer/deserializer functions. 
We have previously developed techniques to improve power efficiency of link building blocks.
In our previous study~\cite{saxena20152}, we utilized charge-based techniques to implement equalization (both continuous-time and decision feedback), serialization and deserialization to improve link power efficiency to about 3mW/Gb/s at 14Gb/s. 
In another study~\cite{elkholy201610}, we recently demonstrated ultra low-power clock generation methods using injection locking. 
At 8GHz, the prototype clock multiplier achieves about $\rm 100 fs_{rms}$-integrated jitter while consuming only 2mW, resulting in more than 5dB improvement in the jitter/power figure-of-merit. 
This low jitter makes such a clock generator suitable for 16Gb/s half-rate link architecture or even 32Gb/s quarter-rate operation. 
So far, neither circuit nor architectural techniques to lower clock distribution power have been explored. 
Further, commonly used dual-loop clock and data recovery  architecture further exacerbates this issue, as it needs multiple clocks with precise phase spacing to perform linear phase interpolation. 
In view of this, we will expand injection-locking techniques that were shown to be power and area efficient for generating low jitter clocks to also implement clock and data recovery. 
We will develop new clocking architectures that minimize clock distribution power so as to improve by the link power efficiency to 1mW/Gb/s while operating at a data rate of at least 20Gb/s.

