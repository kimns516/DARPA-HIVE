\noindent
\textbf{Columbia University.} 
Seok maintains a 600 sq-ft laboratory with 11 stations for designing VLSI hardware, and 5 stations for testing VLSI hardware. 
The available instruments include a high-precision current-meters, desktop multi-meter, five workstations equipped with National Instrument (NI) Labview digital/analog acquisition/actuation boards, programmable -40/+140oC environmental chambers, a professional soldering station and multiple programmable power supplies. 
Seok has accesses to a variety of software including NI Academic Site License for automated testing; industrial EDA tools such as H-SPICE, DesignCompiler, PrimeTime. %which include Synopsis H-SPICE, DesignCompiler, PowerCompiler, PrimeTime, PrimePower, Mentor-Graphics Eldo, Calibre, Cadence Virtuoso, Spectre, Encounter/Velocity, ELC; the nanometer VLSI technology design kits from 180nm down to 28nm nodes; and standard core cell, IO cells, and bonding pad libraries for most of those technologies for design automation. 
%Seok maintains 7 high performance Linux computing servers (each server has 4 to 24 cores and 16-32GB RAM), 9 Linux/Windows workstations, and an 8TB file-server. 

\vspace{3pt}
\noindent
\textbf{University of Illinois.}
Kim, Hanumolu, and Hwu are affiliated with Coordinated Science Laboratory (CSL). 
They all have access to world class super computing and cluster computing facilities to run massive number of parallel simulations. 
%maintains a cluster of 12 Intel Xeon-based dual-socket servers and a cluster of 24 AMD APU-based single-socket servers, 
%connected by a stack of high-speed Ethernet switches linked to the campus network backbone with a 100Gb/s uplink to the CSL facility protected behind the campus firewall.
%Kim also requests two servers to strengthen the cluster as some servers need to be retired due to their age. 
%These two clusters provide more than 300 CPU cores, facilitating fast exploration of various system architectures, simulation of a system designed at the register-transfer and gate levels, and physical design of chips for the project.
% Pavan - can you write about the libraries, tools, and measurement equipment, etc.?
%The hardware lab at the University of Illinois is a characterization facility that consists of 
They also maintain computer-controlled instrumentation tools including DC to RF cascade probe stations, device characterization equipment, 
mixed-signal circuit characterization equipment (RF signal sources, data acquisition, arbitrary waveform generators, etc.), RF devices and circuit characterization equipment (26 GHz spectrum analyzer, 20 GHz network analyzer), and communications test equipment capable of generating almost all digital baseband and RF modulation schemes and performing bit-error-rate and communication channel measurements up to 26GHz. 
%Kim, Hanumolu, and Hwu have accesses to the state-of-the-art circuit design, analysis, and synthesis tools from Cadence and Synopsys.

\vspace{3pt}
\noindent
\textbf{University of Southern California.}
Annavaram has a simulation pool server rack consisting of about 50 servers (200 cores and about 4GB of memory per core). 
%The server pool is fully configured for auto job scheduling and remote management. 
%The proposed work will use and develop a variety of software simulation tools which will be simulated on the pool. 
%Annavaram also requests two servers to strengthen the pool as some servers need to be retired due to their age. 
The lab also currently hosts a set of Xilinx ML-509 FPGA boards that use the Virtex-5 FPGAs which may be used to design and debug the corelets. 
%He has also access to a separate cluster of compute nodes that can run all Xilinx design tools. 
