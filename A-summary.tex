\noindent

Graph processing is a critical component in a broad range of applications ranging from  social network analysis, medical diagnostics, and drug interaction studies. Given the vast size of these graphs with billions of edges and vertices, graph analytics will likely become the most challenging workload to execute efficiently for computing systems in the near future. While there are multiple contending approaches to process graphs, in this proposal we treat graph analytics as linear algebraic problem formulations. This approach represents graphs as adjacency matrices, and many well known graph algorithms are transformed into matrix operations. The evolving duality between graph algorithms and their equivalent matrix operations forms a solid basis for designing computing systems that can efficiently manipulate matrix operations. For example, there are well known linear algebraic problem formulations for graph algorithms such as breadth first search, strongly connected components, shortest paths, stress centrality, betweenness centrality,  sub-graph detection and so on. Given such a wide range of graph algorithms there is an equally diverse range of matrix manipulations that must be supported in hardware for efficiently solving their corresponding linear algebraic formulation. These operations include matrix multiplication, Kronecker product,  matrix decomposition, inner and outer products, matrix inversion, matrix transpose and Eigen value computation. Supporting such a broad range of operations on extremely large matrices is the first challenge that will be tackled in this proposal. 

An equally daunting challenge is the representation of large graphs as adjacency matrices in memory. Large graphs are extremely sparse and there is strong evidence that many large graphs today exhibit power law distributions. For instance, only a few vertices in a graph have very high edge count. Several representations have been developed to improve the storage efficiency of sparse graphs. For instance, double compressed sparse row and column (DCSR and DCSC) representations have been proposed for space efficiency. But  such space efficient representations lead to highly irregular memory access streams. Compressed sparse row and column representations provide more predictable access streams but they provide lower compression ratio thereby requiring more data to be accessed for the same computation. In all these cases memory access bandwidth often becomes the primary performance bottleneck.  While processing-in-memory (PIM) offers high bandwidth, memory access patterns in graph analytics make existing PIM based acceleration schemes ineffective. 

%The difficulty is further compounded by the fact that different sparse representations are suited for different kinds of graphs. A given graph connectivity which translates into a corresponding sparsity distribution prefers one data representation over the other.  While the data representation is the responsibility of the application programmer it is also imperative for the hardware to map the preferred data representation to the underlying memory hierarchy  is stored. The utility of a graph for answering particular questions is usually dictated by the vertex/edge schema. The more specific the schema, the better it will be at addressing a specific question. 

Finally,  scaling the execution of linear algebraic manipulations through parallelization  is the last challenge that this proposal tackles. Parallel execution of matrix operations can be mapped to various distributed computing paradigms, such as Mapreduce, Pregel and Parallel-R. In these distributed paradigms each iteration of a operation, such as a single iteration of LU decomposition, is mapped to multiple nodes and when the iteration completes a new distributed task may be launched for the second iteration.   Almost all these paradigms divide the sparse matrix data across multiple compute nodes. For instance, many PGAS (partitioned global address space) approaches distribute equal number of vertices across nodes (equal number of rows from the adjacency matrix). But when vertex degrees exhibit power law distribution it  is extremely difficult to balance the load across computational nodes. For instance, parallel performance of a mapreduce algorithm will be determined the long tail distribution of the map tasks.  Tackling these imbalances require dynamically remapping data during different iterations. Such remapping schemes demand high bandwidth overheads. Launching a  new iteration of the computation traditionally requires the previous iteration to complete, thereby forcing a barrier synchronization across various threads. In the presence of load imbalances such synchronizations curtail scaling. Hence, new hardware based schemes must be explored to provide stochastic approaches to iterative computing.    

%After the map jobs complete then shuffling the required data across multiple compute nodes is necessary to initiate the reduction process. But data shuffling costs can be traded with additional computational costs at 
 
%of matrices over many compute nodes. Decomposition of array into parallel regions. Map layout 

%Block rows with overlap PGAS provides some data to be overlapped across multiple regions. Could be used for coded map reduce options if we are to explore that.

%PGAS remapping of data for dynamic data load distribution equally.

%On the algorithmic front, graph analytics require significant interconnect bandwidth and synchronization support leading to massive slowdowns and load imbalances during computations. 
These observations lead to the main research thrusts of this proposal, as described below.

First, we propose to design an accelerator architecture that is ideally suited for matrix manipulations. The accelerator core will be architected with a coarse-grain reconfigurable array of execution units so that operations for a complex graph primitive function can be executed by a single instruction and memory access in a data-flow manner within an accelerator. To efficiently support streaming graph analytics, we propose �updated buffer�� which allows an accelerator core to apply computations and changes to a batch of vertices rather than individual vertices. % NAM: SAY ANYTHING MORE ABOUT THE ACCELERATOR PIPELINE   

In order to fully utilize the accelerator pipeline we must tackle the challenge of supplying matrix data as input to the accelerator pipeline. The bandwidth demands for this approach far exceed traditional DRAM interfaces. Hence, we propose to explore recently available high bandwidth memory interfaces.  But when matrices are represented in compressed format the resulting accesses are highly irregular.  The difficulty is further compounded by the fact that different sparse representations are suited for different kinds of graphs. A given graph connectivity which translates into a corresponding sparsity distribution prefers one data representation over the other.  While the data representation is the responsibility of the application programmer it is also imperative for the hardware to understand the semantics of data representation and use that information for data access. We propose to explore processing near high bandwidth memory option to enable the processing layer to understand the semantics of data representation and use appropriate (pre)fetching and caching mechanisms.   High bandwidth memory (HBM) and hybrid memory cube (HMC) used for recent PIM architectures can provide high bandwidth only for sequential, regular memory accesses. In contrast, graph analytics require efficient support for irregular and small granularity memory accesses. We tackle this problem by adopting a recent DiRAM4 3D Memory from Tezzaron and architecting on-chip caches and a custom memory controller to efficiently support small granularity irregular memory accesses. 
Specifically, DiRAM4 provides the same aggregate bandwidth as HBM2, but it provides 4X more (but narrower) channels than HBM2.  Exploiting this design, we propose to design a new cache architecture that efficiently supports both fine- and coarse-grained data accesses with short latency through spatial voltage boosting. We will architect the DiRAM4 controller to fully exploit many narrow memory channels, to fit the graph analytics (e.g., provid-ed data-map) for more efficient bandwidth utilization.

As described earlier, scalability of parallel graph algorithms is bottlenecked by the need to shuffle data after  map tasks complete their local computations.  But data shuffling costs can be traded with additional computational costs by allowing each map task to potentially do extra computations. There are known approaches to coded computation that have been developed recently that may be used to reduce the communication costs. In addition, it has been shown previously that some of the synchronization costs can be eliminated by allowing distributed graph processing tasks to operate on partially completed computations or incomplete inputs. We propose a hardware unit to efficiently support such relaxed synchronization techniques and thus reduce synchronization overhead. 

Apart from relaxed synchronization scalability can also be enhanced by predicting data access patterns. We will design a trailblazer threads to exploit the repeated computations across multiple vertices to learn about data access patterns and opportunistically bring anticipated data from other PIM packages or the main memory of the host system.

Lastly, a PIM package integrating HBM, HMC, and even DiRAM4 with accelerators can provide only limited memory capacity per channel as they use point-to-point interconnects.  For large graphs this requires frequent data transfers from/to the large main memory of the host system and/or other PIM packages.  This demands many high-bandwidth links, posing power, thermal, and signal integrity challenges.  We tackle this challenge with various charge-based and injection-locking techniques for high-speed links.

%Two kinds of graph patterns: static and temporal. Static include degree distribution, scree plot (eigenvalues vs. rank), distribution of components of principal eigenvector. Temporal include diameter over time and densification power law.

%map the preferred data representation to the underlying memory hierarchy  is stored. The utility of a graph for answering particular questions is usually dictated by the vertex/edge schema. The more specific the schema, the better it will be at addressing a specific question. 
