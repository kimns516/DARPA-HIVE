%1. A general description of the objective (for each defined task/activity);
%2. A detailed description of the approach to be taken to accomplish each defined task/activity;
%3. Identification of the primary organization responsible for task execution (prime, sub, team member, by name, etc.);
%4. The completion criteria for each task/activity - a product, event or milestone that defines its completion.
%5. Define all deliverables (reporting, data, reports, software, etc.) to be provided to the Government in support of the proposed research tasks/activities; AND
%6. Clearly identify any tasks/subtasks (prime or subcontracted) that will be accomplished on-campus at a university.
\noindent

\subsection{Phase 1}
\noindent
\textbf{(Task-1) ISA definition of the GAMA corelet} 
 Annavaram  will define a subset of the processor's instruction set dealing with computational aspects of matrix operations. Annavaram with them implement the functional simulator of the corelet in software to implement these instructions.
This task is completed when the functional simulator can run test code and deliver the following items: 
(1) the completed software code for the functional simulator of computational aspects with the test code; and (2) technical documents depicting the computational components of the instruction set, architecture diagrams, and simulator implementation with the simulator usage.


\vspace{3pt}
\noindent
\textbf{Model (Task-2) GAMA corelet's processing pipeline}
Annavaram will model the detailed microarchitecture and performance of GAMA corelet pipeline, which includes (1) Coarse-grain reconfigurable execution pipeline and (2) Elastic cache design private to each corelet; (2) Elastic eDRAM cache shared by all corelets; 
The task is completed, when the software code from the tasks can successfully run test code derived from given benchmarks; provide performance in terms of cycles; and deliver the following items:
(1) the completed performance modeling of corelet software; and (2) technical documents depicting microarchitecture diagrams and performance simulator implementation with the simulator usage.

\vspace{3pt}
\noindent
\textbf{(Task-7) Microarchitecture Simulator of GAMA system scaled to 16 GAMA tiles.}
 Annavaram will integrate processor model into a larger simulation infrastructure to create an entire GAMA tile and measure its performance in software. He will then integrate computational components from 16 tiles to create the GAMA system simulator. 
This task is completed when we the performance simulator can successfully run test code and  provide performance in terms of simulated cycles. 16 tiles will be simulated as a GAMA system. We will deliver the following items:
(1) the completed performance simulator software code with the test code; and (2) technical documents depicting architecture diagrams and performance simulator implementation with the simulator usage.


\subsection{Phase 2}
%- (X) Implement and refine chip architecture (based on TA2 feedback from simulator) to address current memory bottlenecks for sparse data acceleration of graph primitives
%- (X) Update dataflow model / architecture scalable to a 16 nodes with anticipated performance numbers
%- (X) Develop system level scaling model including electrical, mechanical, and thermal needs
%
%-Output:
% (X) Updated block diagram and schematic of graph processor
% (X) PCB/FPGA prototype of chip and performance results on graph benchmarks /primitives (for TA2 to use for toolkit development)
% (X) GDS2 file in preparation for fabrication

\vspace{3pt}
\noindent
\textbf{Implement (Task-8) GAMA corelet pipeline, and (Task-11) GAMA tile integration.}
 Annavaram will perform the RTL designs of the processing pipeline in Verilog HDL; synthesize the RTL designs to generate gate-level netlists; and test the synthesized netlists using a gate-level simulator and a DNVUF2\_HPC\_PCIE FPGA development board.  The RTL design will be integrated into the entire proposed accelerator module in Task 11. 
These tasks are completed when the synthesized netlists can successfuly run the test code on a DNVUF2\_HPC\_PCIE FPGA development board, and deliver the following items:
(1) the completed Verilog HDL code with synthesized gate-level netlists, synthesis scripts, and simulation scripts for each task; and 
(2) technical document describing the RTL design and expected operating frequency of the accelerator module in the target 28nm technology.


\vspace{3pt}
\noindent
\textbf{(Task-12) Iterative refinement of GAMA tile design.} 
 Annavaram will refine the processor pipeline architecture of the accelerator module, the microarchitectures of the accelerator and its sub-systems, and the RTL designs, after analyzing the performance of the accelerator module with the architectural simulator refined by the synthesis results.
This task is completed when the architecture and microarchitectures are improved based on the simulations using the architectural simulator of the corelet pipeline; 
the RTL designs are updated accordingly; and 
deliver the following items:
(1) the refined RTL designs, synthesized gate-level netlists, and synthesis scripts; 
(2) the enhanced architectural simulator software code; and
(3) technical document describing the enhanced architecture and microarchitectures with updated performance projections.

\vspace{3pt}
\noindent
\textbf{(Task-17) FPGA Prototype of GAMA tile}
 Annavaram will prototype a single acclerator module's processor pipeline using DNVUF2\_HPC\_PCIE boards that can be connected to a host system through the PCIe interface.
This task is completed when the host system supplies the benchmark code and its input data, run the code with the input data on DNVUF2\_HPC\_PCIE FPGA development boards prototyping the accelerator module, and deliver the following items:
(1) the completed Verilog HDL code and scripts used for prototyping the accelerator module,  
(2) technical document describing the prototyped implementation and projected performance results in the target 28nm technology.

\subsection{Phase 3}
%-Bring up and test chip fab
%-Identify any bugs that can be worked around with software
%-Identify any bugs which require a chip re-spin
%-Build and test system at scale (16 nodes)
%-Output: Deliver test board with silicon and scale system (16 nodes) to TA3 for testing of toolkits

\vspace{3pt}
\noindent
\textbf{(Task-20) Build and bring up the GAMA system.}
Annavaram will put together the processor chips into the whole system by integrating the accelerator system into PCB, and IBM P8 system.
This task is completed when the full accelerator system runs the test code, and deliver the full accelerator system.



