Our team is comprised of five PIs with circuit, architecture, programming model, and compiler expertise. 
While we are not pursuing TA2, one PI (Hwu) is also an expert in programming models and compilers for GPUs and graph analytics. 
Thus we anticipate a speedier path to incorporate innovations from TA2 into our architecture design and syner-gistically work with other TA2 teams.

\noindent
\textbf{Kim} (\url{http://icsl.ece.illinois.edu}) is an IEEE Fellow and a professor at the University of Illinois. 
He has (co)led many projects sponsored by DARPA, NSF, BAE Systems, IBM, Samsung, AMD, and Microsoft in the past. 
He has actively published many papers on innovative high-performance processor and memory sub-systems architectures, 
His recent studies on memory system architecture near-DRAM processing has been published to the top architecture conferences (ISCA'14, HPCA'15, ISCA'15, MICRO'15, HPCA'16, MICRO'16).
%~\cite{nda:hpca:kim,nda:ieeemicro:kim,nda:micro:kim}
Beside, Kim has co-developed an architecture-level GPU power model (dubbed GPUWattch) and an architecture-level CPU+GPU processor simulator to evaluate performance and power efficiency benefits of various proposed techniques [70, 72]. 
To date, GPUWattch has been downloaded by more than 1300 users and the paper has been cited by more than 200 other scholarly articles according to Google Scholar. 
He is a member of in the IEEE HPCA Hall of Fame and IEEE MICRO Hall of Fame. Lastly, he designed state-of-the-art experimental chips, all of which were published in ISSCC, VLSI Circuit Symposium, and JSSC, during his tenure at Intel.

\noindent
\textbf{Hwu:} (\url{http://impact.crhc.illinois.edu}) is the Walter J. (``Jerry'') Sanders III-Advanced Micro Devices Endowed Chair at the University of Illinois. 
His research expertise include architecture, implementation, software for high-performance computer systems and parallel processing with GPUs. 
He is the PI and co-director of many large-scale research projects and centers including the pet-ascale Blue Water system, the Intel and Microsoft funded Universal Parallel Computing Re-search, the world’s first NIVIDA GPU Center of Excellence, and IBM Center for Cognitive Computing. 
He has received many prestigious awards including ACM Maurice Wilkes Award in 1998, ACM Grace Murray Hopper Award in 1999. He is a Fellow of IEEE and ACM.

\noindent
\textbf{Hanumolu:} (\url{}) is an Associate Professor at the University of Illinois. 
His research expertize is in the areas of energy-efficient integrated circuit implementation of analog-digital signal processing circuits, sensor interfaces, wireline communication systems, and power-conversion. 
He has led and is currently leading many projects sponsored by DARPA, Semiconductor Research Corporation (SRC), NSF, Intel, IBM, Broadcom, Qualcomm, and Texas Instruments among others. 
He guided the design of more than 50 research prototype integrated circuits and the results from which are published in venues such as ISSCC and JSSC and frequently employed in products as well.   

\noindent
\textbf{Annavaram:} (\url{http://scip-lab.usc.edu}) is the Robert G. and Mary G. Lane early career chair professor at the University of Southern California. 
During his career he designed and demonstrated several physical prototype architectures for energy efficient computing. 
He co-developed a die-stacked microarchitecture design of an Intel mi-croprocessor which was the first published architecture research work on 3D stacking. 
While working at Intel his research on Energy Per Instruction (EPI) throttling, formed the foundation for TurboBoost technology.

\noindent
\textbf{Seok:} (\url{http://ee.columbia.edu/~mgseok}) is a professor at Columbia University, conducting research on various aspects of digital, analog, mixed-signal, and VLSI system-on-chip design and prototyping. 
His expertise includes energy-efficient embedded memory and pipeline designs. 
In these areas, Seok has (co)led more than 20 chip prototyping projects, sponsored by DARPA, NSF, Samsung, SKHynix, and other institutions in the past. 
The results of the projects were published in more than 70 articles in the top conferences and journals such as ISSCC, VLSI, CICC, JSSC, TCAS-I, TVLSI, etc and have advanced state of the arts in computer hardware designs. 
Among those, the results that are the most relevant to this project are: ultra-low-power embedded SRAM design with temporal speed-up ability that contributes the most energy-efficient and the most compact FFT core (ISSCC’17); 
and an in-situ error detection and correction technique for massively-parallel, specialized architectures using spatially and temporarily fine-grained voltage boosting and body swapping techniques (VLSI’14, JSSC’15, VLSI’16, ISSCC’17). 
