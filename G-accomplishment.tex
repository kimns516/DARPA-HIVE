Our team is comprised of five PIs with circuit, architecture, programming model, and compiler expertise. 
While we are not pursuing TA2, one PI (Hwu) is also an expert in programming models and compilers for GPUs and graph analytics. 
Thus we anticipate a speedier path to incorporate innovations from TA2 into our architecture design and syner-gistically work with other TA2 teams.

\noindent
\textbf{Kim} \url{http://icsl.ece.illinois.edu} is an IEEE Fellow and a professor at the University of Illinois. 
His research expertise include architecture and circuits for high-performance processors and on/off-chip memories. 
He has actively investigated innovative high-performance processor and memory sub-systems architectures, publishing more than 140 papers to top-tier computer architecture conferences with more than 6000 citations according to Google Scholar. 
%His recent work on near-DRAM processing has been published to the top architecture conferences~\cite{nda:hpca:kim,nda:ieeemicro:kim,nda:micro:kim}
Beside, Kim has co-developed an architecture-level GPU power model (dubbed GPUWattch) and an architecture-level CPU+GPU processor simulator to evaluate performance and power efficiency benefits of various proposed techniques [70, 72]. 
To date, GPUWattch has been downloaded by more than 1300 users and the paper has been cited by more than 200 other scholarly articles according to Google Scholar. 
He is a member of in the IEEE HPCA Hall of Fame and IEEE MICRO Hall of Fame. Lastly, he designed state-of-the-art experimental chips, all of which were published in ISSCC, VLSI Circuit Symposium, and JSSC, during his tenure at Intel.
He has (co)led many projects sponsored by DARPA, NSF, BAE Systems, IBM, Samsung, AMD, and Microsoft in the past. 

\noindent
\textbf{Hwu:} \url{http://impact.crhc.illinois.edu} is the Walter J. (``Jerry'') Sanders III-Advanced Micro Devices Endowed Chair at the University of Illinois. 
His research expertise include architecture, implementation, software for high-performance computer systems and parallel processing with GPUs. 
He is the PI and co-director of many large-scale research projects and centers including the pet-ascale Blue Water system, the Intel and Microsoft funded Universal Parallel Computing Re-search, the world’s first NIVIDA GPU Center of Excellence, and IBM Center for Cognitive Computing. 
He has received many prestigious awards including ACM Maurice Wilkes Award in 1998, ACM Grace Murray Hopper Award in 1999. He is a Fellow of IEEE and ACM.

\noindent
\textbf{Hanumolu:} is an Associate Professor at the University of Illinois. 
His research expertize is in the areas of energy-efficient integrated circuit implementation of analog-digital signal processing circuits, sensor interfaces, wireline communication systems, and power-conversion. 
He has led and is currently leading many projects sponsored by DARPA, Semiconductor Research Corporation (SRC), NSF, Intel, IBM, Broadcom, Qualcomm, Texas Instruments among others. 
He guided the design of more than 50 research prototype integrated circuits and the results from which are published in venues such as ISSCC and JSSC and frequently employed in products as well.   

\noindent
\textbf{Annavaram:}

\noindent
\textbf{Seok:}

